%\VignetteIndexEntry{abpm Overview}
%\VignetteDepends{abpm }
%\VignetteKeywords{abpm }
%\VignetteKeywords{abpm }
%\VignetteKeywords{abpm }
%\VignettePackage{abpm }
\documentclass[a4paper]{article}

\newcommand{\Rfunction}[1]{{\texttt{#1}}}
\newcommand{\Robject}[1]{{\texttt{#1}}}
\newcommand{\Rpackage}[1]{{\textit{#1}}}
\newcommand{\Rclass}[1]{{\textit{#1}}}
\newcommand{\Rmethod}[1]{{\textit{#1}}}

\author{Xiaoyong Sun & DavidA James}

\usepackage{/CHBS/apps/R/2.7.0/lib64/R/share/texmf/Sweave}
\begin{document}

\setkeys{Gin}{width=1\textwidth} 

\title{Quick Guide for abpm Package}
\maketitle

\tableofcontents
%%%%%%%%%%%%%%%%%%%%%%%%%%%%%%%%%%%%%%%%%%%%%%

\section{Introduction}
Ambulatory blood pressure measurement (ABPM) is a monitoring tool widely used 
in the research of cardiovascular diseases and drug development.  
Patients typically wear a portable device for one or more days collecting BP readings
 about every 10-30 minutes during their daily routines. 
 This process is then repeated at longer intervals (weeks or months) to assess, 
 for instance, the effect of therapies on BP. A number of advantages of ABPM over 
 traditional single BP measurement have been reported in the literature, and they 
 include the reduction of the so-called "white-coat" bias [1-3], the assessment of 
 nocturnal BP lowering (so-called "dipping") [3-4], quantification of pharmacological 
 effects over time [5], distortion of BP circadian rhythms, etc.  
\section{Data managment section}
This section includes data cleaning and data preprocessing. Data cleaning is to clean ABPM data
which is more than 24 hours after dosing, and data preprocessing is to translate date
object to numeric.

\begin{verbatim}
date<-abpm.date(s.abp$ABPDBP1N,s.abp$ABPSBP1N, s.abp$ABPSMD1O, s.abp$REA1O, time.abs=T)
data<-cbind(s.abp, date)
my.abp<-abpm.clean(data)
\end{verbatim}

There is also a function to deal with multiple visit: abpm.change. It can
calculate change among multiple visit.
\begin{verbatim}
t<-abpm.change(done.abp$DIFF, done.abp$ABPDBP1N, done.abp$STYSID1A, done.abp$TRTN, done.abp$DOSE, 1)
\end{verbatim}

\section{Model section}
This package provides three modeling methods: cosinor model(parametric method),
loess and local likelihood(nonparametric method).
\newline
\newline
For cosinor model, you can choose single cosinor, double cosinor or triple cosinor,
\begin{verbatim}
d1<-done.abp[done.abp$TRTN==1 & done.abp$DOSE==1,]
fit.single<-abpm.model(d1$DIFF, d1$ABPDBP1N, d1$STYSID1A, d1$TRTN, d1$DOSE, 
		method=3, nlme=1, cosinor.period=c(24))
fit.double<-abpm.model(d1$DIFF, d1$ABPDBP1N, d1$STYSID1A, d1$TRTN, d1$DOSE, 
		method=3, nlme=1, cosinor.period=c(24, 12))
fit.triple<-abpm.model(d1$DIFF, d1$ABPDBP1N, d1$STYSID1A, d1$TRTN, d1$DOSE, 
		method=3, nlme=1, cosinor.period=c(24, 12, 6))
\end{verbatim}
\newline
\newline
Sometime, you may have large data set, and nlme package does not converge well. You can
switch to lme4 package instead with a option: nlme=2.
\begin{verbatim}
fit.lme4<-abpm.model(d1$DIFF, d1$ABPDBP1N, d1$STYSID1A, d1$TRTN, d1$DOSE, 
		method=3, nlme=2, cosinor.period=c(24, 12, 6))

\end{verbatim}
The option: cosinor.period is to use control circadian rhythm. for ABPM data,
we found triple cosinor fits data quite well.
\newline
\section{Feature section}
In this section, 24 hour mean, daytime/nighttime mean, morning surge, smoothness index
and trough:peak ratio is calculated.
\begin{verbatim}
f2<-abpm.feature(done.abp$DIFF, done.abp$ABPDBP1N, done.abp$STYSID1A, done.abp$TRTN, done.abp$DOSE, method="hour")
\end{verbatim}
There are a lot of options for this function, such as morning surge, day time, 
dipper ratio, peak time, and trough time, etc. Users can define it with
regards to their own protocols.

\section{Subgroup section}
We have define resonder/nonresponder, high daytime/nightime blood pressure and dipper/non-dipper
to discern various subgroups. Users can check percent of certain subgroup against threshhold values.
\begin{verbatim}
f2<-abpm.feature(done.abp$DIFF, done.abp$ABPDBP1N, done.abp$STYSID1A, done.abp$TRTN, done.abp$DOSE, method="hour")
\end{verbatim}
\newline
\newline
Users can also visualize these subgroups, and details are demonstrated in next section.

\section{Display section}
This section includes the following functions:
display.ind, display.pop, display.model, display.allmodel and display.percent
\newline
\newline
display.ind is to display individual values, such as blood pressure.
\begin{verbatim}
display.ind(done.abp$DIFF, done.abp$ABPDBP1N, done.abp$STYSID1A, 
done.abp$TRTN, done.abp$DOSE, night=c(20,25), type=c("n", "smooth"), main="test",
xlab="test", span=0.25
)
\end{verbatim}
display.pop is to visualize population based on loess.
\begin{verbatim}
display.pop(done.abp$DIFF, done.abp$ABPDBP1N, done.abp$STYSID1A, done.abp$TRTN, 
 done.abp$DOSE, night=c(20,25), ylim=c(60,120), layout=c(1,3), page=T)

\end{verbatim}
display.model is to visualize model fit with data.
\begin{verbatim}
fit1<-abpm.model(d1$DIFF, d1$ABPDBP1N, d1$STYSID1A, d1$TRTN, d1$DOSE, 
		method=1, span=0.25)
display.model(fit1,d1$DIFF, d1$ABPDBP1N, d1$STYSID1A,
visit=d1$DOSE)
\end{verbatim}
display.allmodel is to visualize all model fit with data.
\begin{verbatim}
fitlist<-list(fit1=fit1, fit2=fit2, fit3=fit3)
display.allmodel(fitlist, d1$DIFF, d1$ABPDBP1N,
 d1$STYSID1A, d1$DOSE)
\end{verbatim}

display.percent is to visualize subgroup.
\begin{verbatim}
proplist<-abpm.percent(done.abp$DIFF, done.abp$ABPDBP1N, done.abp$STYSID1A, done.abp$TRTN,done.abp$DOSE)
display.percent(proplist, page=T)

\end{verbatim}


 

\end{document}




